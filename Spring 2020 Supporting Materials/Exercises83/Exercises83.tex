
\documentclass[12pt]{article}
\usepackage{fancybox}
\usepackage{fullpage}
\usepackage{psfig}
\usepackage{url}
%\newcommand{\onef}[3]{\psfig{figure=#1,width=#2,height=#3}}
\usepackage{xcolor}
\usepackage{color, colortbl}
%\definecolor{lightgray}{gray}{0.9}

\newcommand{\onef}[3]{
\begin{center}~
{\psfig{figure=#1,width=#2,height=#3}}
\vskip -0.15truein
\end{center}}

%___
%--------------------------------------------
\newcommand{\twof}[4]
{
\hbox to\hsize{\hss
    \vbox{\psfig{figure=#1,width=#3,height=#4}}\qquad
    \vbox{\psfig{figure=#2,width=#3,height=#4}}
    \hss}
}


%---------------------------------
\usepackage[framemethod=TikZ]{mdframed}
\usepackage{lipsum}
\mdfdefinestyle{MyFrame}{%
    linecolor=black,
    outerlinewidth=1.5pt,
    roundcorner=5pt,
    innertopmargin=\baselineskip,
    innerbottommargin=\baselineskip,
    innerrightmargin=20pt,
    innerleftmargin=20pt,
    backgroundcolor=gray!20!white}


 \newcommand{\be}{\begin{eqnarray}}
\newcommand{\ee}{\end{eqnarray}}
\newcommand{\ba}{\begin{eqnarray*}}
\newcommand{\ea}{\end{eqnarray*}}




\begin{document}
\thispagestyle{empty}
%\addtocounter{page}{-1}
\vspace*{-0.1in}

\begin{center}
        { \Large \bf   8.3 EXERCISES}
\end{center}
\noindent {\bf BMED6420}

\noindent  Brani  Vidakovic;  Fall 2018


\noindent Consult the class slides, hints, and  cited literature for
the solution of exercise problems.

%missing, censoring
\vspace{0.5in}



\vspace*{0.2in}
\noindent  {\bf 1.  Dukes'  C
Colorectal Cancer and Diet Treatment.~} %===========================================
%
 Colorectal cancer is a common cause of death. In the advanced
stage of disease, when the disease is first diagnosed in many patients,
surgery is the only  treatment. Cytotoxic drugs,
when given as an adjunct to surgery,  do not prevent
relapse and do not increase the survival
in patients with advanced disease.

 Interest has been shown, at least by patients, in a nutritional
approach to treatment,
where diet plays a critical role in the disease management program.


 In a controlled clinical trial, McIllmurray  and  Turkie (1987)
  evaluated the diet treatment in patients with Dukes' C
colorectal cancer, because the residual tumour mass is small after operation,
the relapse rate is high, and no other effective treatment is available.
The diet treatment consisted of  linolenic
acid, an oil extract of the seed from
the evening primrose plant  {\it Onagraceae Oenothera biennis}   and vitamin E.

The data for the treatment  and control  patients are given below:

\begin{center}
\begin{tabular}{ll}
\hline
Treatment	& Survival time (months)\\
\hline
  Linoleic acid  &	1+, 5+, 6, 6, 9+, 10, 10, 10+, 12, 12, 12, 12, 12+, 13+, 15+,\\
 ($n_1=25$)    & 16+, 20+, 24, 24+, 27+, 32, 34+, 36+, 36+, 44+\\
  \hline
Control  &	3+, 6, 6, 6, 6, 8, 8, 12, 12, 12+, 15+, 16+, 18+, 18+, 20, 22+,\\
 ($n_2=24$)   & 24, 28+, 28+, 28+, 30, 30+, 33+, 42\\
\hline
\end{tabular}
\end{center}


 Fit the data with Weibull distribution, taking the treatment/control (1/0) as a covariate.
 Place noninformative priors on all parameters. Consult the file {\tt Leukemia.odc} from the repository.
 Is the linoleic acid treatment beneficial? Comment.




\vspace*{0.2in}
\noindent  {\bf 2.  Censored Rayleigh.}

%
 The lifetime (in hours) of a certain sensor has
Rayleigh distribution, with survival function
\ba
S(t) = \exp\left\{ -\frac{1}{2} \lambda t^2\right\},~\lambda > 0.
\ea
Twelve sensors are placed under test for 100 hours, and the following
failure times are recorded
23, 40, 41, 67, 69, 72, 84, 84, 88, 100+, 100+.
Here + denotes a censored time.

(a) If failure times $t_1, \dots, t_r$ are observed, and $t_{r+1}^+, \dots, t_{n}^+$
are censored,  find the Bayes estimator of $\lambda$ . Use noninformative gamma prior on $\lambda.$


(b) Evaluate $S(t)$ for $t=60$ and find 95\% Credible Set.

(c) The MLE for $\lambda$ is
\ba
\hat \lambda = \frac{ 2 r }{\sum_{i=1}^r t_i + \sum_{i=r+1}^n t^{+2}_i }.
\ea
Evaluate the MLE for the given data and comment on closeness to the Bayes estimator in (a).

\vspace*{0.2in}
\noindent {Hint:} Rayleigh distribution is not implemented as default in
WinBUGS/OpenBUGS and has to be dealt using zero-tricks. Additional complication is censoring.
Here is an example how to do censoring on distributions defined by
zero-trick. In the set of variables $y_1, \dots, y_9$ from normal distribution
with mean $\mu$ and variance 1, the last three values
are censored at $ t=8.$

The standard representation with censoring
\begin{verbatim}
model {
  for (i in 1:6) {y[i] ~ dnorm(mu, 1)}        # uncensored data
  for (i in 7:9) {y[i] ~ dnorm(mu, 1)I(8,)}   # censored data
  mu                   ~ dunif(0, 100)
}

Data:
list(y = c(6,6,6,7,7,7,NA,NA,NA))

 node	 mean	 sd	 MC error	2.5%	median	97.5%	start	sample
	mu	7.193	0.3478	0.003604	6.515	7.19	7.875	1001	10000
\end{verbatim}

is equivalent to

\begin{verbatim}
model {
  for (i in 1:6) {y[i] ~ dnorm(mu, 1)}
  for (i in 1:3) {
    zeros[i]          <- 0
    zeros[i]           ~ dpois(p[i])
    p[i]              <- -log(phi(mu-8))
    #each censored observation provides term P(Y-mu>8-mu) to
    #the likelihood of mu, which is equal to 1-phi(8-mu)=phi(mu-8),
    # for phi being the cdf of standard normal
  }
  mu                   ~ dunif(0, 100)
}

Data:
list(y = c(6,6,6,7,7,7,NA,NA,NA))

	 node	 mean	 sd	 MC error	2.5%	median	97.5%	start	sample
	mu	7.192	0.348	0.003687	6.519	7.185	7.882	1001	10000
\end{verbatim}




\vspace*{0.2in}
\noindent  {\bf 3.  Stagnant Water with MAR Data.}

Carlin et al (1992)\footnote{Hierarchical Bayesian Analysis of Changepoint Problems
Bradley P. Carlin, Alan E. Gelfand and Adrian F. M. Smith
Journal of the Royal Statistical Society. Series C (Applied Statistics)
Vol. 41, No. 2 (1992), pp. 389-405.}
analyse data on stagnation of water by  piecing together
linear parametric forms.

$\bullet $  $y_i$ is log flow rate down the channel.

$\bullet $  $x_i$ is log height of stagnant surface levels for 
different surfactants  $i$.

Proposed model is
\ba
y_i &\sim& {\cal N}(\mu_i, \sigma^2)\\
\mu_i &=& \alpha + \beta_1 \cdot x_i + \beta_2 \cdot (x_i - \theta)_+
\ea
Here $a_+$ is $a$ if $a \geq 0$ and 0 if $a < 0$.

According to this model, regression slope is $\beta_1$ for
$  x < \theta$ 
and $\beta_1 + \beta_2$ for $ x \geq \theta.$

The original exercise is modified to have two $y$'s and two $x$'s missing 
at random.



\begin{verbatim}
model {
  for (i in 1:N) {
    y[i]    ~ dnorm(mu[i], tau)
    mu[i]  <- alpha + beta[1]*x[i] + beta[2]*(x[i] - theta)
            * step(x[i] - theta)
  }
  for( i in 1:N) {
      x[i] ~ dunif(-5,5)}
  tau       ~ dgamma(0.001, 0.001)
  alpha     ~ dnorm(0.0, 1.0E-6)
  for (j in 1:2) {
    beta[j] ~ dnorm(0.0, 1.0E-6)
  }
  sigma    <- 1/sqrt(tau)
  theta     ~ dunif(-1.3, 1.1)
}
	
Data
list(y = c(1.12, 1.12, 0.99, 1.03, 0.92, NA, 0.81, 0.83, 0.65, 0.67, 0.60,  
0.59, 0.51, 0.44, 0.43, 0.43, 0.33, 0.30, 0.25, NA, 0.13, -0.01, -0.13, 
 -0.14, -0.30, -0.33, -0.46,  -0.43, -0.65),
     x = c(-1.39, -1.39, -1.08, -1.08, -0.94, -0.80, -0.63, -0.63, -0.25, -0.25, 
     -0.12, NA, 0.01, 0.11, 0.11, 0.11,  0.25, 0.25, 0.34, 0.34, 0.44, 0.59, 
     0.70, 0.70, 0.85, NA,  0.99, 0.99, 1.19),
     N = 29)


Inits
list(alpha = 0.2, beta = c(-0.45, 0), tau = 5, theta = 0)

+ Gen Inits

Complete data results:
	 node	 mean	 sd	 MC error	2.5%	median	97.5%	start	sample
	alpha	0.5482	0.01337	4.934E-4	0.5228	0.5475	0.5756	501	10000
	beta[1]	-0.4187	0.01555	5.097E-4	-0.4485	-0.4193	-0.3869	501	10000
	beta[2]	-0.5944	0.0212	2.248E-4	-0.6362	-0.5942	-0.5528	501	10000
	sigma	0.02218	0.003352	5.138E-5	0.01673	0.02178	0.02995	501	10000
	theta	0.02563	0.03378	0.001399	-0.04016	0.02783	0.08707	501	10000
	
Omitted from the original data  y_6=0.90,  y_20 = 0.24, x_12=-0.12, x_26=0.85

	mean	sd	MC_error	val2.5pc	median	val97.5pc	start	sample
	alpha	0.5498	0.01519	2.87E-4	0.5204	0.5496	0.5794	1001	100000
	beta[1]	-0.4153	0.01676	3.064E-4	-0.4485	-0.4153	-0.3826	1001	100000
	beta[2]	-0.5892	0.02205	2.957E-4	-0.6323	-0.5895	-0.5453	1001	100000
	sigma	0.02197	0.00356	2.742E-5	0.01632	0.02151	0.03022	1001	100000
	theta	0.01631	0.03713	7.755E-4	-0.05168	0.01715	0.08537	1001	100000
	x[12]	-0.1016	0.05593	5.299E-4	-0.2168	-0.09895	-0.002997	1001	100000
	x[26]	0.8855	0.0236	1.313E-4	0.839	0.8854	0.9327	1001	100000
	y[6]	0.8821	0.02345	1.156E-4	0.8356	0.8821	0.9281	1001	100000
	y[20]	0.2179	0.02347	1.358E-4	0.1715	0.2179	0.2644	1001	100000
\end{verbatim}

\end{document}
