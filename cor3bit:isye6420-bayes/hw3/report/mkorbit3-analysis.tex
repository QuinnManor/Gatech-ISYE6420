\documentclass[a4 paper]{article}
% Set target color model to RGB



\usepackage[inner=2.0cm,outer=2.0cm,top=2.5cm,bottom=2.5cm]{geometry}
\usepackage{setspace}
\usepackage[rgb]{xcolor}
\usepackage{verbatim}
\usepackage{subcaption}
\usepackage{amsgen,amsmath,amstext,amsbsy,amsopn,tikz,amssymb,tkz-linknodes}
\usepackage{fancyhdr}
\usepackage[colorlinks=true, urlcolor=blue,  linkcolor=blue, citecolor=blue]{hyperref}
\usepackage[colorinlistoftodos]{todonotes}
\usepackage{rotating}
%\usetikzlibrary{through,backgrounds}

\usepackage{lmodern}
\usepackage[T1]{fontenc}
\usepackage[capposition=top]{floatrow}
\usepackage{hyperref}
\usepackage{graphicx}
\graphicspath{ {images/} }
\usepackage{booktabs}
\usepackage{changepage}
\usepackage{float}








\hypersetup{%
pdfauthor={Nick Korbit},%
pdftitle={Homework},%
%pdfkeywords={Tikz,latex,bootstrap,uncertaintes},%
pdfcreator={PDFLaTeX},%
pdfproducer={PDFLaTeX},%
}
%\usetikzlibrary{shadows}
% \usepackage[francais]{babel}
\usepackage{booktabs}


\newcommand{\ra}[1]{\renewcommand{\arraystretch}{#1}}

\newtheorem{thm}{Theorem}[section]
\newtheorem{prop}[thm]{Proposition}
\newtheorem{lem}[thm]{Lemma}
\newtheorem{cor}[thm]{Corollary}
\newtheorem{defn}[thm]{Definition}
\newtheorem{rem}[thm]{Remark}
\numberwithin{equation}{section}

\newcommand{\homework}[6]{
   \pagestyle{myheadings}
   \thispagestyle{plain}
   \newpage
   \setcounter{page}{1}
   \noindent
   \begin{center}
   \framebox{
      \vbox{\vspace{2mm}
    \hbox to 6.28in { {\bf ISYE 6420:~Bayesian Statistics \hfill {\small #2}} }
       \vspace{6mm}
       \hbox to 6.28in { {\Large \hfill #1  \hfill} }
       \vspace{6mm}
       \hbox to 6.28in { {\it Instructor: {\rm #3} \hfill Name: {\rm #5}, gtID: {\rm #6}} }
       %\hbox to 6.28in { {\it TA: #4  \hfill #6}}
      \vspace{2mm}}
   }
   \end{center}
   \markboth{#5 -- #1}{#5 -- #1}
   \vspace*{4mm}
}

\newcommand{\problem}[2]{~\\\fbox{\textbf{Problem #1}}\newline\newline}
\newcommand{\subproblem}[1]{~\newline\textbf{(#1)}}
\newcommand{\D}{\mathcal{D}}
\newcommand{\Hy}{\mathcal{H}}
\newcommand{\VS}{\textrm{VS}}
\newcommand{\solution}{~\newline\textbf{\textit{(Solution)}} }

\newcommand{\bbF}{\mathbb{F}}
\newcommand{\bbX}{\mathbb{X}}
\newcommand{\bI}{\mathbf{I}}
\newcommand{\bX}{\mathbf{X}}
\newcommand{\bY}{\mathbf{Y}}
\newcommand{\bepsilon}{\boldsymbol{\epsilon}}
\newcommand{\balpha}{\boldsymbol{\alpha}}
\newcommand{\bbeta}{\boldsymbol{\beta}}
\newcommand{\0}{\mathbf{0}}



%%%%%%%%%%%%%%%%%%%%%%%%%%%%%%%%%%%%%%%%
%%			 Document				  %%
%%%%%%%%%%%%%%%%%%%%%%%%%%%%%%%%%%%%%%%%


\begin{document}
	
\homework{Homework \#3}{Spring 2020}{Roshan Vengazhiyil, Brani Vidakovic}{}{Nick Korbit}{903263968}


%%%%%%%%%%%%%%%%%%%%%%%%%%%%%%%%%%%%%%%%
%%			 Problem 1				  %%
%%%%%%%%%%%%%%%%%%%%%%%%%%%%%%%%%%%%%%%%




\problem{1}

\textbf{a$)$.} Let's start with MLE estimate.
Using Poisson pmf $\frac{\theta^{x} e^{-\theta}}{x !}$
we first calculate the likelihood $f(x|\theta)$
based on our experiment data:

$$
f(x|\theta) = \frac{\theta^{1}e^{-\theta}}{1!}\times\frac{\theta^{2}e^{-\theta}}{2!}\times\frac{\theta^{0}e^{-\theta}}{0!}\times\frac{\theta^{1}e^{-\theta}}{1!}=e^{-4\theta}\left(\theta^{1}\frac{\theta^{2}}{2}\theta^{1}\right)=\frac{e^{-4\theta}\theta^{4}}{2}
$$

In order to find MLE estimate we need to maximize 
likelihood $f(x|\theta)$:
$$
\hat{\theta}_{MLE}=argmax_{\theta}\frac{e^{-4\theta}\theta^{4}}{2}
$$

We simplify as follows:
$$
\hat{\theta}_{MLE}=argmax_{\theta}e^{-4\theta}\theta^{4}=argmax{}_{\theta}\ln\left(e^{-4\theta}\theta^{4}\right)=argmax_{\theta}\left(4ln\theta-4\theta\right)
$$

$$
d/d\theta\left(4ln\theta-4\theta\right)=4/\theta-4;d/d\theta^{2}\left(4ln\theta-\theta\right)=-4/\theta^{2}
$$

At $\theta=1$ $d/d\theta=0$ and $;d/d\theta^{2}<0$, so we 
conclude that  $\theta=1$ is a maximum. So that 
$$
\hat{\theta}_{MLE}=1
$$

We now turn to Bayes estimator $\hat{\theta}_b$:
$$
\hat{\theta}_{b}\propto f(x|\theta)\pi(\theta)
$$

For our experiment that means
$$
\hat{\theta}_{b}\propto\frac{e^{-4\theta}\theta^{4}}{2}\theta^{-1/2}
$$
$$
\hat{\theta}_{b}\propto e^{-4\theta}\theta^{3.5}
$$

We can recognize Gamma distribution 
with parameters $\alpha=4.5$ and $\beta=4$:
$$
\hat{\theta}_{b}\propto \mathcal{G}a\left(4.5,4\right)
$$

The mean of Gamma is $\alpha/\beta$, so that 
our Bayes estimator becomes:

$$
\hat{\theta}_{b}=\frac{4.5}{4}=1.125
$$






\textbf{b$)$.} We start with the lower bound $L$:
$$
\int_{-\infty}^{L} \pi(\theta | x) d \theta=\alpha / 2
$$
We can substitute the integral with the Gamma cdf:
$$
F_{X}(L) =\alpha / 2
$$
We solve for $L$ numerically using the Brent solver 
implemented via scipy \textit{root\_scalar} function. 
$$
F_{X}(L) - 0.05 / 2 = 0; L\approx 0.3375 
$$

Let's turn to the upper bound $U$:
$$
\int_{-\infty}^{U} \pi(\theta | x) d \theta= 1 - \alpha / 2
$$

Alternatively
$$
F_{X}(U) = 1 - \alpha / 2
$$

Solving for $U$ gives us:
$$
F_{X}(U) - 1 + 0.05 / 2 = 0; U \approx 2.3778
$$

Resulting in the credible set 
$[0.3375, 2.3778]$ of length $l=2.0403$. \newline 


\textbf{c$)$.} In order to find HPD credible set 
we are essentially optimizing for the shortest 
credible set. Let's translate the problem into the
``optimization language'' and solve it 
with scipy \textit{optimize} package:

We minimize the function $f(L,U)$:
$$
f(L,U) = U-L
$$
Subject to constraints:
$$
U>L
$$
$$
F_{X}(U)-F_{X}(L)\geq1-\alpha
$$

After providing an initial guess $L=0$ and $U=3$ and
solving with scipy we obtain
the HPD credible set $[0.23782339, 2.17403328]$.
The length of the HPD set 
is $l \approx 1.9362$. \newline



\textbf{d$)$.} To find the MAP estimator we need to 
calculate the mode of the posterior.
We calculated previously 

$$
\pi(\theta|x)\propto e^{-4\theta}\theta^{3.5}
$$

The mode of $\pi(\theta|x)$ is a point 
where the posterior peaks. To find the mode
we therefore need to 
calculate $argmax_{\theta}\pi(\theta|x)$,
essentially solving

$$
argmax_{\theta}e^{-4\theta}\theta^{3.5}
$$

Let's run the scipy minimize procedure:

$$
\hat{\theta}_{MAP}\approx0.874999997
$$

Now we check $\hat{\theta}_{MAP}$
with the analytical value --
mode of the $\mathcal{G}a\left(4.5,4\right)$:

$$
mode=\frac{\alpha-1}{\beta}
$$
$$
mode=\frac{4.5-1}{4}=3.5/4 = 0.875 
$$

We see that, indeed, $$\hat{\theta}_{MAP}=0.875$$ \newline


\textbf{e$)$.} Let's start with $p_0$ and $p_1$:

$$
\begin{array}{l}
{p_{0}=\int_{\Theta_{0}} \pi(\theta | x) d \theta=\mathbb{P}^{\theta | X}\left(H_{0}\right)} \\
{p_{1}=\int_{\Theta_{1}} \pi(\theta | x) d \theta=\mathbb{P}^{\theta | X}\left(H_{1}\right)}
\end{array}
$$

We use posterior $\mathcal{G}a\left(4.5,4\right)$ cdf 
to calculate $p_0$ and $p_1$:
$$
p_{1}=F_{X}(1.0)\approx0.46585
$$
$$
p_{0}=1-p_{1};p_{0}\approx0.53415
$$

So, based on the posterior (and the 
fact that our Jeffreys' prior is non-informative)
we would prefer hypothesis $H_0$. \newline


Note. The script for solving Q1 is implemented in 
\textit{hw3.py}, function \textit{solve\_q1()} 
(included in the zip archive). To run the code just run 
\textit{'python hw3.py'}. 


%%%%%%%%%%%%%%%%%%%%%%%%%%%%%%%%%%%%%%%%
%%			 Problem 2				  %%
%%%%%%%%%%%%%%%%%%%%%%%%%%%%%%%%%%%%%%%%


\problem{2}

\textbf{a$)$.} Let's start with posterior probabilities 
$p0$ and $p1$. We know that

$$
p_{0}=P\left(H_{0} | X\right)=\left[1+\frac{\pi_{1}}{\pi_{0}} \cdot \frac{m_{1}(x)}{f(x | 0.5)}\right]^{-1}
$$

$$
f(x | p)=\left(\begin{array}{l}
{n} \\
{x}
\end{array}\right) p^{x}(1-p)^{n-x}
$$

$$
m_{1}(x)=\int_{0.5}^{1} f(x | p) 2 d p
$$

Substituting with our values we get
$$
f(x|0.5)=\left(\begin{array}{l}
16\\
15
\end{array}\right)0.5^{15}0.5^{1}=\left(\begin{array}{l}
16\\
15
\end{array}\right)0.5^{16}\approx 0.00024414
$$


$$
m_{1}(x)=\int_{0.5}^{1}\left(\begin{array}{l}
16\\
15
\end{array}\right)p^{15}(1-p)^{1}2dp=32\int_{0.5}^{1}p^{15}(1-p)^{1}dp=32\int_{0.5}^{1}\left(p^{15}-p^{16}\right)dp
$$

$$
m_{1}(x)=32\times\left.\frac{p^{16}}{16}-\frac{p^{17}}{17}\right|_{0.5}^{1}=32\times\left(\left(\frac{1}{16}-\frac{1}{17}\right)-\left(\frac{0.5^{16}}{16}-\frac{0.5^{17}}{17}\right)\right)\approx0.1176309
$$

$$
p_{0}=\left[1+\frac{0.05}{0.95}\cdot\frac{0.1176309}{0.00024414}\right]^{-1}\approx0.037938
$$

$$
p_1 = 1 - p_0
$$

$$
p_1 \approx 1 - 0.037938 \approx 0.962062
$$


We now calculate the Bayes factors:
$$
B_{01}=\frac{f(x | 0.5)}{m_{1}(x)}
$$

$$
B_{01}\approx\frac{0.00024414}{0.1176309}\approx0.0020755
$$

$$
B_{10}=\frac{1}{B_{01}}
$$

$$
B_{10}=\frac{1}{0.0020755} \approx 481.816
$$



\textbf{b$)$.} Let's calibrate the BF for $H_1$: $log_{10}B_{10}\approx2.68288$.
According to the Jeffreys' scale 
we have decisive evidence against 
$H_0$ ($\log _{10} B_{10}(x)>2$). \newline

Note. The script for solving Q2 is implemented in 
\textit{hw3.py}, function \textit{solve\_q2()} 
(included in the zip archive). To run the code just run 
\textit{'python hw3.py'}. 




%%%%%%%%%%%%%%%%%%%%%%%%%%%%%%%%%%%%%%%%
%%			 Bibliography			  %%
%%%%%%%%%%%%%%%%%%%%%%%%%%%%%%%%%%%%%%%%

\begin{thebibliography}{9}


\bibitem{stat}\label{stat} 
Engineering Biostatistics: An Introduction using MATLAB and WinBUGS. 
Brani Vidakovic - Wiley Series in Probability and Statistics.

\end{thebibliography}



\end{document} 